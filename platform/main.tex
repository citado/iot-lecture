\documentclass{../iot-lecture}

\subtitle{IoT Platforms}

\begin{document}

\begin{frame}
  \titlepage{}
\end{frame}
\begin{frame}
  \frametitle{Outline}
  \tableofcontents{}
\end{frame}

\section{IoT Platforms, What \& Why?}

\begin{frame}
  \frametitle{Challenges}
  \begin{itemize}
    \item Lack of a common date format and sharing standard.
    \begin{itemize}
      \item Two different sensors can monitor the same parameter using different unit of measure.
    \end{itemize}
    \item Heterogeneity in networking and sensor technologies.
  \end{itemize}
\end{frame}

\begin{frame}
  \frametitle{What is an IoT platform?}
  \begin{block}{Internet of Things Wiki}
    In simple words the purpose of any IoT device is to connect with
    other IoT devices and applications (cloud-based mostly) to relay
    information using internet transfer protocols.
    The gap between the device sensors and data networks is filled
    by an \textit{\color{Green} IoT Platform}.
  \end{block}
\end{frame}

\begin{frame}
  \frametitle{What is their job?}
  \begin{itemize}
    \item Developers can develop their applications with ease
    \item End-users can see his/her dashboard and customize it
    \item The government can provide regulation on data
  \end{itemize}
\end{frame}

\begin{frame}
  \frametitle{Platforms' Components/Services}
  \begin{block}{}
    There is \textbf{no specific} standard for platform architecture.
    Here we use a de-facto standard for having a better arrangement of materials.
  \end{block}
  \begin{itemize}
    \item Layer 1
    \item Layer 2
    \item Layer 3
  \end{itemize}
\end{frame}

\end{document}
