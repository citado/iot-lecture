\documentclass{../iot-lecture}

\subtitle{IoT Platforms}

\addbibresource{main.bib}

\begin{document}

\begin{frame}
  \titlepage{}
\end{frame}
\begin{frame}
  \frametitle{Outline}
  \tableofcontents{}
\end{frame}

\section{IoT Platforms, What \& Why?}

\begin{frame}
  \frametitle{Challenges\footfullcite{Rub2020}}
  \begin{itemize}
    \item Lack of a common date format and sharing standard.
    \begin{itemize}
      \item Two different sensors can monitor the same parameter using different unit of measure.
    \end{itemize}
    \item Heterogeneity in networking and sensor technologies.
    \begin{itemize}
      \item Heterogeneous environment of IoT devices/plaforms that must be integrated into an interoperable one.
    \end{itemize}
  \end{itemize}
\end{frame}

\begin{frame}
  \frametitle{Challenges (Cont'd)}
  \begin{itemize}
    \item Lack of a standardized definition of environmental indicators.
    \begin{itemize}
      \item Standard doesn't consider clear metrics for water quality (i.e., temperature, pH, conductivity, and dissolved oxygen indicators)
    \end{itemize}
    \item Semantic interoperability between IoT solutions for the environment domain.
    \begin{itemize}
      \item Environmental studies can comprehend many phenomena through the observation/analysis of different features
        (i.e., earthquake can be predicted by vibrations or satellite image processing).
      \item A platform for environment studies must model and explore the semantic relation between heterogeneous data sets.
    \end{itemize}
  \end{itemize}
\end{frame}

\begin{frame}
  \frametitle{What is an IoT platform?}
  \begin{block}{Internet of Things Wiki\footfullcite{Top20IoTPlatforms}}
    In simple words the purpose of any \textbf{\color{YellowOrange} IoT device} is to connect with
    other IoT devices and applications (cloud-based mostly) to relay
    information using internet transfer protocols.

    The gap between the device sensors and data networks is filled
    by an \textit{\color{Green} IoT Platform}.
  \end{block}
\end{frame}

\begin{frame}
  \frametitle{What is their job?}
  \begin{itemize}
    \item Developers can develop their applications with ease
    \item End-users can see his/her dashboard and customize it
    \item The government can provide regulation on data
  \end{itemize}
\end{frame}

\begin{frame}
  \frametitle{Platforms' Components/Services}
  \begin{block}{}
    There is \textbf{no specific} standard for platform architecture.
    Here we use a de-facto standard for having a better arrangement of materials.
  \end{block}
  \begin{itemize}
    \item Layer 1
    \item Layer 2
    \item Layer 3
  \end{itemize}
\end{frame}

%\begin{frame}[allowframebreaks]
%  \printbibliography%
%\end{frame}

\end{document}
