\documentclass[]{report}

\begin{document}

\قسمت{دریافت داده‌ها}

عموما پلتفرم‌های اینترنت اشیا داده‌ها را در یک ساختار مشخص
و با استفاده از تعدادی پروتکل از پیش تعیین شده دریافت می‌کنند.
استانداردهایی مانند \متن‌لاتین{SenML} برای ساختار داده‌ها می‌تواند
کمک کننده باشد و از سوی دیگر پروتکل‌های ارتباطی مانند \متن‌لاتین{MQTT}
و یا \متن‌لاتین{HTTP} برای ارتباط شناخته شده هستند.

در صورتی که بخواهیم داده‌ها را از شبکه‌هایی غیر از اینترنت مانند \متن‌لاتین{LoRaWAN}
دریافت کنیم می‌بایست یک کامپوننت دیگر توسعه داده که داده‌ها را دریافت و آن‌ها به ساختار
داده‌ای پلتفرم تغییر دهد.

\قسمت{مدل داده‌ای}

برای اشیا و توانایی‌های مرتبط با آن‌ها نیاز به تعریف داریم. این تعریف می‌بایست توسط کاربر صورت
گرفته و به پلتفرم داده شود. برای این قسمت \متن‌لاتین{OneDM} کاربری خواهد بود.

\قسمت{ترکیب داده‌ها}

یکی از نیازمندی‌های پلتفرم تجمیع داده‌ها و استفاده از نتیجه آن است.
برای پیاده‌سازی این قسمت می‌توان یک \متن‌لاتین{Rule Engine} داشت و جریان‌های داده‌ای
را به واسطه آن پردازش کرد.

\قسمت{پیدا کردن سرویس‌ها}

\قسمت{ارتباط دستگاه به دستگاه}

\قسمت{قابلیت اجرا در لبه}

\end{document}
