\documentclass[]{./reports}

\title{نیازمندی‌های میان‌افزار در اینترنت اشیا}

\begin{document}

\عنوان‌ساز

به صورت کلی، یک میان‌افزار پیچدگی‌های سیستم یا سخت‌افزار را انتزاغی کرده و اجازه می‌دهد
توسعه‌دهندگان برنامه‌های کاربردی همه تلاش خود را روی کاری که باید انجام شود متمرکز کنند بدون
اینکه دغدغه‌هایی در سطح سیستم یا سخت‌افزار داشته باشند.
در اینترنت اشیا، تنوع زیادی در تکنولوژی‌های ارتباطی مورد استفاده و تکنولوژی‌های سطح سیستم وجود دارد که
میان‌افزار می‌بایست از هر دو دیدگاه تا حد لازم پشتیبانی کند.
نیازمندی‌های میان‌افزار برای پشتیبانی از اینترنت اشیا را می‌توان به دو دسته نیازمندی‌های کارکردی و غیرکارکردی تقسیم کرد.
نیازمندی‌های کارکردی شامل سرویس‌ها و کارکردها می‌شوند و نیازمندی‌های غیرکارکردی شامل پشتیبانی از کیفیت سرویس
و مشکلات کارایی می‌شوند.

\قسمت{مدیریت داده‌ها}

داده‌ها کلید برنامه‌های اینترنت اشیا هستند. در اینترنت اشیا داده‌ها عموما به داده‌های حسگرها
یا هر داده‌ای از شبکه‌ی زیرساخت که برای برنامه‌های کاربردی اهمیت داشته باشد گفته می‌شود.
میان‌افزار اینترنت اشیا نیاز دارد تا سرویس‌های مدیریت داده شامل جمع‌آوری داده، پردازش داده و ذخیره‌سازی
آن را برای برنامه‌های کاربردی فراهم آورد.
پردازش و پیش‌پردازش می‌تواند شامل فیلتر کردن، فشرده‌سازی و ترکیب باشد.

\زیرقسمت{دریافت داده‌ها}

عموما پلتفرم‌های اینترنت اشیا داده‌ها را در یک ساختار مشخص
و با استفاده از تعدادی پروتکل از پیش تعیین شده دریافت می‌کنند.
استانداردهایی مانند \متن‌لاتین{SenML} برای ساختار داده‌ها می‌تواند
کمک کننده باشد و از سوی دیگر پروتکل‌های ارتباطی مانند \متن‌لاتین{MQTT}
و یا \متن‌لاتین{HTTP} برای ارتباط شناخته شده هستند.

در صورتی که بخواهیم داده‌ها را از شبکه‌هایی غیر از اینترنت مانند \متن‌لاتین{LoRaWAN}
دریافت کنیم می‌بایست یک کامپوننت دیگر توسعه داده که داده‌ها را دریافت و آن‌ها به ساختار
داده‌ای پلتفرم تغییر دهد.

\زیرقسمت{مدل داده‌ای}

برای اشیا و توانایی‌های مرتبط با آن‌ها نیاز به تعریف داریم. این تعریف می‌بایست توسط کاربر صورت
گرفته و به پلتفرم داده شود. برای این قسمت \متن‌لاتین{OneDM} کاربری خواهد بود.

\زیرقسمت{ترکیب داده‌ها}

یکی از نیازمندی‌های پلتفرم تجمیع داده‌ها و استفاده از نتیجه آن است.
برای پیاده‌سازی این قسمت می‌توان یک \متن‌لاتین{Rule Engine} داشت و جریان‌های داده‌ای
را به واسطه آن پردازش کرد.

\قسمت{پیدا کردن سرویس‌ها و منابع}

یکی از قابلیت‌های مهم در پلتفرم‌ها توانایی شناسایی اشیا موجود، سرویس‌های ارائه شده توسط آن‌ها و وضعیت سلامتی‌شان است.
برای اینکار یکی نیازمندی اصلی وجود تعریف از شی و قابلیت‌های آن است.
مساله مهم دیگر زمان‌بندی استفاده از اشیا است. با توجه به وضعیت سلامتی یک شی و درخواست‌های موجود می‌بایست اشیا
به طرز بهینه‌ای تخصیص و زمان‌بندی شوند.

از آنجایی که محیط و زیرساخت اینترنت اشیا پویا است، فرض‌های مربتط با دانش کلی یا قطعی در رابطه با دسترسی‌پذیری این
منابع نادرست هستند.
از آنجایی که دخالت انسانی در پیدا کردن منابع ناممکن است، پیدا کردن سرویس‌ها می‌بایست به صورت خودکار صورت بگیرد.

\قسمت{مدیریت منابع}

کیفیت سرویس قابل قبولی برای همه برنامه‌های کاربردی مورد انتظار است و در محیطی مانند محیط اینترنت اشیا که منابع تاثیرگذار بر کیفیت سرویس
محدود هستند، مهم است که سرویسی برای برنامه‌ها در جهت مدیریت این منابع فراهم شود.

\قسمت{مدیریت رویدادها}

به صورت بالقوه تعداد بسیار زیادی رویداد در برنامه‌های کاربردی اینترنت اشیا تولید می‌شوند
که می‌بایست به واسطه یک قسمت داخلی از میان‌افزار اینترنت اشیا مدیریت شوند.
مدیریت رویدادها یک رویداد مشاهده‌ای ساده را به یک رویداد با معنی تبدیل می‌کند.

\قسمت{ارتباط دستگاه به دستگاه}

\قسمت{قابلیت اجرا در لبه}

\پایان‌ساز

\end{document}
